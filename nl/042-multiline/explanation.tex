\paragraph{Meer afhankelijke instructies}
In het volgende voorbeeld zie je dat niet \'{e}\'{e}n maar twee instructies afhankelijk zijn van de \texttt{if}:

\begin{verbatim}
1  a = 0
2  if(a < 10)
3      a = a + 10
4      a = a * 2
\end{verbatim}

De trace is vergelijkbaar met eerder:

\begin{tracelist-left}[l|ccccccc]
regel & \texttt{1} & \texttt{2} & 3 & 4 \\ \hline
a & \fbox{\texttt{0}} & \texttt{0} & \fbox{10} & \fbox{20}
\end{tracelist-left}

\paragraph{Meerdere variabelen}
Nu introduceren we weer meerdere variabelen. Het gevolg is dat we meer informatie moeten bijhouden bij het traceren. Kijk bijvoorbeeld naar het volgende programma:

\begin{verbatim}
1  a = 1
2  b = 0
3  if(a > b)
4      c = a
5      a = b
6      b = c
\end{verbatim}

De trace ziet er dan als volgt uit, met voor elke variabele een rij:

\begin{tracelist-left}[l|ccccccc]
regel & \texttt{1} & \texttt{2} & \texttt{3} &  \texttt{4} & \texttt{5} &  \texttt{6} \\ \hline
\texttt{a} & \fbox{\texttt{1}} & \texttt{1} & \texttt{1} & \texttt{1} & \fbox{\texttt{0}} & \texttt{0} \\
\texttt{b} & & \fbox{\texttt{0}} & \texttt{0} & \texttt{0} & \texttt{0} & \fbox{\texttt{1}} \\
\texttt{c} & & & & \fbox{\texttt{1}} & \texttt{1} & \texttt{1} \\
\end{tracelist-left}
