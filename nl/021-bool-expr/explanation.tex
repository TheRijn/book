\paragraph{Beweringen zijn waar of onwaar}

Van beweringen zoals \texttt{4\,==\,5} kunnen we bepalen of ze \emph{waar} zijn. Het getal \texttt{4} is niet gelijk aan \texttt{5}, dus de bewering is \emph{onwaar}. Er zijn zes belangrijke manieren om een bewering te doen:

\begin{center}
\begin{tabular}{ll}
\textbf{operatie} &        \textbf{betekenis}\\
\midrule
      \texttt{==} &               is gelijk aan\\
      \texttt{!=} &          is niet gelijk aan\\
       \texttt{<} &              is kleiner dan\\
       \texttt{>} &               is groter dan\\
      \texttt{<=} & is kleiner dan of gelijk aan\\
      \texttt{>=} & is groter dan of gelijk aan\\
\midrule
\end{tabular}
\end{center}

Uit al deze operaties kunnen twee waarden komen: \texttt{true} (waar) en \texttt{false} (onwaar). Deze waardes worden ook wel \emph{booleans} genoemd\footnote{Genoemd naar George Boole, die als eerste een algebraïsch systeem voor logica ontwikkelde halverwege de 19e eeuw.}. De operaties waaruit deze booleans komen, worden ook wel \emph{boolean expressions} genoemd.

\paragraph{Prioriteitsregels}

Net als bij rekenkundige operaties gelden prioriteitsregels bij het evalueren van beweringen:

\begin{enumerate}
  \item eerst de expressies tussen haakjes
  \item dan rekenkundige operaties zoals \texttt{*} en \texttt{+} (let op de regels!)
  \item dan \texttt{>}, \texttt{<}, \texttt{==}, \texttt{!=}, \texttt{>=}, \texttt{<=}
\end{enumerate}


\paragraph{Verschillende soorten getallen}

Je vindt in beweringen zowel gehele getallen als floating-pointgetallen. Als een geheel getal wordt vergeleken met een floating-pointgetal, vindt automatische conversie plaats, waarbij van de \texttt{int} een \texttt{float} wordt gemaakt.
