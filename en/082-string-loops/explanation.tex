\paragraph{Strings with loops}

By using loops we can extract information from strings. For example, we can write a loop that prints only some of the characters from a string:

\begin{nnflisting}
s = "Hsopil"
i = 0
len = length(s)
while(i < len)
    print(s[i])
    i = i + 2
\end{nnflisting}

On line \texttt{3} we use the \texttt{length} function to determine the size or length of a string \texttt{s} (the result is 6 in this case). Most programming languages have such a built-in command.

As soon as we are going to trace a code fragment like the above, it is useful to annotate the positions in the string, because we will need to \emph{look up} the characters by position quite often.

\begin{tabular}{l|ccccccccccccc}
index&0&1&2&3&4&5\\ \hline
karakter&H&s&o&p&i&l
\end{tabular}

\paragraph{Tracing}

The key parts to keep track of when tracing this code are the variable \texttt{i}, the character at position \texttt{i} in the string. We will also want to track the \emph{moments} that the print statement occurs, so we can eventually see what will be printed by the program.

To keep the trace as small as possible, we will not note the values of the variables that do not change: the string \texttt{s} itself, and the length of the string, as stored in the variable \texttt{len}.

\setlength\tabcolsep{5pt}
\begin{tracelist-left}[l|cccccccccccccccccccccccccccccccccccccccc]
regel & \texttt{1} & \texttt{2} & \texttt{3} &
        \texttt{4} & \texttt{5} & \texttt{6} &
				\texttt{4} & \texttt{5} & \texttt{6} &
				\texttt{4} & \texttt{5} & \texttt{6} &
				\texttt{4}  \\ \hline
var \texttt{i} &   & \fbox{0} & 0 &
                 0 & 0 & \fbox{2} &
								 2 & 2 & \fbox{4} &
								 4 & 4 & \fbox{6} &
								 6 \\
s[i]           &   & H & H &
               H & H & o &
               o & o & i &
               i & i  &
               & \\
print          &&&&
               &H&&
							 &o&&
							 &i&&
							 & \\
\end{tracelist-left}
\setlength\tabcolsep{6pt}

We can now conclude that the code will print the letters \texttt{Hoi} to our screens.
