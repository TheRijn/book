\paragraph{Strings} A string is composed of several \emph{symbols}, also called \emph{characters}. Let's define a string \texttt{s} that comprises 13 characters:

\begin{verbatim}
s = "Hello, world!"
\end{verbatim}

The characters in such a string have a \emph{position} or \emph{index}. In most programming languages, the index is 0-bases, or in other words, we start counting at zero.

\begin{tabular}{l|ccccccccccccc}
character&\texttt{H}&\texttt{e}&\texttt{l}&\texttt{l}&\texttt{o}&\texttt{,}&\texttt{ }&\texttt{w}&\texttt{o}&\texttt{r}&\texttt{l}&\texttt{d}&\texttt{!}\\
\hline
index&0&1&2&3&4& 5&6&7&8&9& 10&11&12 \\
\end{tabular}

\paragraph{Indexing}

We can retrieve separate characters from a string by \emph{indexing into} it:

\begin{minipage}[t]{0.5\textwidth}
\begin{listing}
s = "Hello, world!"
print(s[1])
\end{listing}
\end{minipage}%
\begin{minipage}[t]{0.5\textwidth}
\begin{listing}
e
\end{listing}
\end{minipage}

This fragment prints the character from a string \texttt{"Hello, world!"} that is found at position 1. Because we count from 0, this is the letter \texttt{e}, one of the characters that computers can represent.

\paragraph{Using variables for indexing}

To allow us to create algorithms that use strings, it's possible to index into a string using variables or more complex expressions:

\begin{minipage}[t]{0.5\textwidth}
\begin{listing}
s = "Hello, world!"
i = 7
print(s[(i - 1) / 2])
\end{listing}
\end{minipage}
\begin{minipage}[t]{0.5\textwidth}
\begin{listing}
l
\end{listing}
\end{minipage}

\paragraph{Boundaries}

Reading or even writing characters outside the \emph{boundaries} or a string will commonly bring about an \emph{out-of-bounds error} or a \emph{segmentation fault}. For example:

\begin{minipage}[t]{0.5\textwidth}
\begin{listing}
s = "Hello, world!"
print(s[20])
\end{listing}
\end{minipage}
\begin{minipage}[t]{0.5\textwidth}
\begin{listing}
error
\end{listing}
\end{minipage}

So keep track of string boundaries!
