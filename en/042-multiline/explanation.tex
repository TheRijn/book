\paragraph{Multiple dependent statements}

In the next code fragment, not one but two statements are dependent on the \texttt{if} condition:

\begin{nnflisting}
a = 0
if(a < 10)
    a = a + 10
    a = a * 2
\end{nnflisting}

The trace is quite like the one we did in the previous section:

\begin{tracelist-left}[lccccccc]
 & \texttt{1} & \texttt{2} & 3 & 4 \\ \hline
\\[-1em]
a & \fbox{\texttt{0}} & \texttt{0} & \fbox{10} & \fbox{20}
\end{tracelist-left}

\paragraph{Multiple variables}

Here we again introduce the use of multiple variables in one program. This means that more information has to be tracked while tracing the program. Take a look at this fragment:

\begin{nnflisting}
a = 1
b = 0
if(a > b)
    c = a
    a = b
    b = c
\end{nnflisting}

To trace this program, we add rows for each variable that is in the program. The variables \texttt{b} and \texttt{c} are not initialized until later, which is why some of the values are blank.

\begin{tracelist-left}[lccccccc]
  & \texttt{1} & \texttt{2} & \texttt{3} &  \texttt{4} & \texttt{5} &  \texttt{6} \\ \hline
\\[-1em]
\texttt{a} & \fbox{\texttt{1}} & \texttt{1} & \texttt{1} & \texttt{1} & \fbox{\texttt{0}} & \texttt{0} \\
\texttt{b} & & \fbox{\texttt{0}} & \texttt{0} & \texttt{0} & \texttt{0} & \fbox{\texttt{1}} \\
\texttt{c} & & & & \fbox{\texttt{1}} & \texttt{1} & \texttt{1} \\
\end{tracelist-left}
