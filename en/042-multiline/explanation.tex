\paragraph{Meer afhankelijke instructies}

In the next code fragment, not one but two instructies are dependent on the \texttt{if} condition:

\begin{verbatim}
1  a = 0
2  if(a < 10)
3      a = a + 10
4      a = a * 2
\end{verbatim}

The trace is quite like the one we did in the previous section:

\begin{tracelist-left}[l|ccccccc]
 & \texttt{1} & \texttt{2} & 3 & 4 \\ \hline
a & \fbox{\texttt{0}} & \texttt{0} & \fbox{10} & \fbox{20}
\end{tracelist-left}

\paragraph{Multiple variables}

Here we again introduce the use of multiple variables in one program. This means that more information has to be tracked while tracing the program. Take a look at this fragment:

\begin{verbatim}
1  a = 1
2  b = 0
3  if(a > b)
4      c = a
5      a = b
6      b = c
\end{verbatim}

To trace this program, we add rows for each variable that is in the program. The variables \texttt{b} and \texttt{c} are not initialized until later, which is why some of the values are blank.

\begin{tracelist-left}[l|ccccccc]
  & \texttt{1} & \texttt{2} & \texttt{3} &  \texttt{4} & \texttt{5} &  \texttt{6} \\ \hline
\texttt{a} & \fbox{\texttt{1}} & \texttt{1} & \texttt{1} & \texttt{1} & \fbox{\texttt{0}} & \texttt{0} \\
\texttt{b} & & \fbox{\texttt{0}} & \texttt{0} & \texttt{0} & \texttt{0} & \fbox{\texttt{1}} \\
\texttt{c} & & & & \fbox{\texttt{1}} & \texttt{1} & \texttt{1} \\
\end{tracelist-left}
