\paragraph{Swapping the values of variables}

We will now see how to combine a number of assignments to develop an \emph{algorithm}. An algorithm is a general procedure that a computer can execute. It is general because it uses a number of variables that may be assigned numbers or other data depending on the situation.

Let's develop an algorithm to swap the values of \underline{two} variables,

\vspace*{-.5\baselineskip}\begin{center}
\texttt{a = 54} \quad and \quad \texttt{b = 29}.
\end{center}\vspace*{-.5\baselineskip}

As a first intuition, you might think to assign the value of \texttt{b} to \texttt{a}, and the value of \texttt{a} to \texttt{b}. An algorithm might then look like this:

\begin{tracelist}
a = b \\
b = a
\end{tracelist}

Let's trace the algorithm. We take two starting values for \texttt{a} and \texttt{b}, and then we show the effect on these two variables \emph{line by line}. Like in the previous section, we use a box to show which variable is \underline{changed} on which line of code.

\begin{tracelist}[l|cc]
      &         a &         b \\
      &        54 &        29 \\
a = b & \fbox{29} &        29 \\
b = a &        29 &  \fbox{29}
\end{tracelist}

Now note that this is not the intended effect of the algorithm! Instead of swapping, the value of \texttt{a} is overwritten with \textbf{b}'s original value, and now both contain the same number. 

The problem is the following: on line three (TODO numbering), we would like to assign \texttt{b} the value of \texttt{a}, but at that point, \texttt{a} has already been overwritten. From this analysis we may have an intuition for a solution: we need to make sure that on line 3, the original value of \texttt{a} is still available. To do this, we need to introduce a third variable, which can ``save'' the value of \texttt{a} before it is overwritten.

\begin{tracelist}
t = a \\
a = b \\
b = t
\end{tracelist}

Tracing that modified program will look like this:

\begin{tracelist}[l|ccc]
      &         a &         b &         t \\
      &        54 &        29 &           \\
t = a &        54 &        29 & \fbox{54} \\
a = b & \fbox{29} &        29 &        54 \\
b = t &        29 & \fbox{54} &        54 \\
\end{tracelist}

Indeed, this is the expected result for swapping!
