\paragraph{Expressions}

An expression is a combination of numbers and operations. Such an expression may be \emph{evaluated} using the mathematical rules that you may already know. For example:

\begin{center}
  \ttfamily
  \begin{tabular}{l@{\qquad}l}
    {\normalfont expression} & {\normalfont evaluates to} \\
    \midrule
    4 & 4 \\
    5  & 5 \\
    4 + 5   & 9 \\
    4 * 5  & 20 \\
    \midrule
  \end{tabular}
\end{center}

Next to the more familiar operations we will explain several exceptions and particulars that come into play when doing calculations using a computer.

\paragraph{Integer division}

Computers treat \emph{integers}, or whole numbers, differently from \emph{floats}, numbers with a decimal point. In most situations you can expect similar results, but there are some differences. The first difference is division: the expression \texttt{1\,/\,2} surprisingly evaluates to \texttt{0}. This is because division of two integers is interpreted as asking the following question: \emph{how often does the number 2 fit into the number 1}? In the following table, we list the results of dividing some numbers by 3.

\begin{center}
  \begin{tabular}{ c @{\hspace{25pt}} l l l l l l l l }
    \texttt{x}    & 0 & 1 & 2 & 3 & 4 & 5 & 6 & 7 \\[.5em]
    \texttt{x\,/\,3}  & 0 & 0 & 0 & 1 & 1 & 1 & 2 & 2 \\
  \end{tabular}
\end{center}

\paragraph{Modulo}

An operation that you might not know is modulo, often written as \texttt{\%}. This operation nicely complements the integer division; for example, in the case of an expression like \texttt{5\,\%\,2}, we should answer the following question: \emph{2 fits 2 times into 5, how much is then left}? The answer is \texttt{1}. In the following table, we list the the results of performing a modulo 3 for some numbers. When you compare it to the previous table, you should see that it does indeed list what is ``left'' after performing an integer division.

\begin{center}
  \begin{tabular}{ c @{\hspace{25pt}} l l l l l l l l }
    \texttt{x}    & 0 & 1 & 2 & 3 & 4 & 5 & 6 & 7 \\[.5em]
    \texttt{x\,\%\,3} & 0 & 1 & 2 & 0 & 1 & 2 & 0 & 1 \\
  \end{tabular}
\end{center}

\paragraph{Divisibility}

We call an integer \emph{divisible} by another integer if the result of taking the modulo is \texttt{0}, or in other words: if nothing is left after the division.
