\paragraph{For-loops}
The code fragment below prints the numbers 0 to 9 using a \texttt{while}-loop.

\begin{nnflisting}
i = 0
while(i < 10)
    print(i)
    i++
\end{nnflisting}

You might have noticed that many loops contain some of the same ingredients. In the program above, line 1 performs \textbf{initialization}, line 2 contains the \textbf{condition} that controls the loop, and line 4 is an \textbf{update} that ensures the loop steps towards the final goal. Because most loops comprise all three components, many programming languages have a special construct to do the same: the \emph{counting for-loop}. The fragment below shows the same program, but instead using a \texttt{for}-loop:

\begin{nnflisting}
for(i = 0; i < 10; i++)
    print(i)
\end{nnflisting}

The same three ingredients, initialization, condition and update, are present in the first line of the loop. By convention, the initialization (\texttt{i = 0}) is performed only once, at the loop's start. The condition (\texttt{i < 0}) is evaluated each time we restart at the top of the loop. The update (\texttt{i++}) is also performed each time, but only \emph{after} all statements in the loop have been performed. So in this example, the \emph{order of execution} is:

\begin{itemize}
\item regel 1, \texttt{i = 0} (initialization)
\item regel 1, \texttt{i < 10} (condition)
\item regel 2, \texttt{print(i)}
\item regel 1, \texttt{i++} (update)
\item regel 1, \texttt{i < 10} (condition)
\item regel 2, \texttt{print(i)}
\item etc.
\end{itemize}

Because the \texttt{for}-loop is a shorthand, it may not be immediately obvious which part is used at which moment. Another way of looking at it, is to put a \texttt{while}-loop and its \texttt{for}-counterpart next to each other:

\begin{tabular}{lp{4em}l}
                                         && \tikzmark{i2}\verb|<init>| \\
\verb|for(<init>|\tikzmark{i1}\verb|; <condition>|\tikzmark{c1}\verb|; <update>|\tikzmark{u1}\verb|)|
&& \verb|while|\tikzmark{c2}\verb|(<condition>)| \\
\verb|    <loop body>|             && \verb|    <loop body>| \\
                                         && \verb|    |\tikzmark{u2}\verb|<update>|
\end{tabular}
\begin{tikzpicture}[overlay,remember picture]
\draw[->, line width=0.5pt] ([yshift=0.9em,xshift=-0.5em]{pic cs:i1}) [bend left] to ([yshift=0.3em]{pic cs:i2});
\draw[->, line width=0.5pt] ([yshift=0em,xshift=-0.5em]{pic cs:c1}) [bend right] to ([yshift=-0.1em]{pic cs:c2});
\draw[->, line width=0.5pt] ([yshift=0em,xshift=-0.5em]{pic cs:u1}) [bend right] to ([yshift=0.3em]{pic cs:u2});
\end{tikzpicture}
