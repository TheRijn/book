\paragraph{Looping}

In the following fragment a variable is changed using a \texttt{while}-loop.

\begin{nnflisting}
a = 0
while(a < 2)
    a = a + 1
a = a / 2
\end{nnflisting}

On line 1, the variable \texttt{a} is initialized with the integer \texttt{0}. On the next line, we see the \texttt{while} keyword, and then between parentheses a condition \texttt{(a < 2)}. Now, \emph{as long as} the condition yields \texttt{true}, any statements within the loop will be repeated. In this case, only line 3 is in the loop, but more statements might be included. It's important to realize that the condition is evaluated \underline{once per loop}: on line 2, each time a decision is made if the loop can be executed again.

\paragraph{Tracing loops}

Like with sequences of assignment statements, we can trace the execution of loops. The technique is a little bit different, because the loop may be executed multiple times. For the previous fragment, the trace looks like this:

\begin{tracelist}[l|ccccccc]
regel & \texttt{1} & \texttt{2} & \texttt{3} &  \texttt{2} &
                          \texttt{3} & \texttt{2} & \texttt{4} \\ \hline
\\[-1em]
var \texttt{a} & \fbox{\texttt{0}} & \texttt{0} & \fbox{\texttt{1}} & \texttt{1} & \fbox{\texttt{2}} & \texttt{2} & \fbox{\texttt{1}}\\
a < 2 & & true &  & true &  & \fbox{false} &
\end{tracelist}

Atop the table are the line numbers for each line that we encounter, from left to right. Below that, we include lines for each variable or expression that we would like to trace. Like before, we write down a value for each variable as it is initialized, and we draw a box around each variable whenever it changes.

We included the condition for the \texttt{while}-loop as part of the trace. We only write the condition on line 2, because it is only calculated and checked on that line. The condition yields a boolean value, so the possible values are \texttt{true} and \texttt{false}.

In this case, the condition is checked three times. The first two times it yields \texttt{true}, so line 3 is run right after. The third time it yields \texttt{false}, after which the rest of the program is run --- in this case only line 4.

% TODO infinite loops would be nice to illustrate the condition check

% \paragraph{Infinite loops}
%
% A condition is usually dependent on the value of one or more variables, like \texttt{a < 2}. To ensure that the loop will end at some point, that variable needs to be changed inside the loop. In the example above, we see that the variable is incremented by 1 each step, meaning that at some point the condition \texttt{a < 2} will not be true anymore.
%
% If the variable in the condition is not changed at all during the loop, the loop can either only run 0 times (if the condition is not met), or it would loop forever (if the condition is met). The simplest way to create an infinite loop is to specify an always-true condition:
%
% \begin{nnflisting}
% a = 0
% while(true)
%     a = a + 1
% a = a / 2
% \end{nnflisting}
%
% In this case, the loop would be ``infinite'', and line 4 would never be reached.
