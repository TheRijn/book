\paragraph{Logic operations}

The values \texttt{true} and \texttt{false}, or any boolean expressions, may be combined into more complex expressions using logic operations. We introduce three.

\paragraph{And}

This operation, written as \texttt{\&\&}, requires both operands (TODO) to be \texttt{true} in order to yield \texttt{true} itself. The behavior of the operator can be described using a truth table:

\begin{center}
  \ttfamily
  \begin{tabular}{r@{ \&\& }l@{\qquad}l}
    \multicolumn{2}{l}{\normalfont expression} & {\normalfont yields} \\
    \midrule
    true & true   & true \\
    true & false & false \\
    false & true  & false \\
    false & false  & false \\
    \midrule
  \end{tabular}
\end{center}

\paragraph{Or}

This operation is written as \texttt{||}, and requires only one of the operands to be \texttt{true} in order to yield \texttt{true}. 

\begin{center}
  \ttfamily
  \begin{tabular}{r@{ || }l@{\qquad}l}
    \multicolumn{2}{l}{\normalfont expression} & {\normalfont yields} \\
    \midrule
    true  & true   & true \\
    true  & false  & true \\
    false & true   & true \\
    false & false  & false \\
    \midrule
  \end{tabular}
\end{center}

\paragraph{Not}

A boolean expression's value can be negated, which is expressed using an exclamation mark \texttt{!}.

\begin{center}
  \ttfamily
  \begin{tabular}{l@{\qquad}l}
    {\normalfont expression} & {\normalfont yields} \\
    \midrule
    !true   & false \\
    !false  & true \\
    \midrule
  \end{tabular}
\end{center}

\paragraph{Precedence}

The logic operations also have their place in the precedence hierarchy:

\begin{enumerate}
  \item first, expressions in parentheses
  \item then arithmetic operations like \texttt{+} en \texttt{*}
  \item then boolean operations like \texttt{>}, \texttt{<}, \texttt{==}, \texttt{!=}, \texttt{>=}, \texttt{<=}
  \item then \texttt{!}
  \item then \texttt{||}
  \item and finally \texttt{\&\&}
\end{enumerate}

(Note that unlike \texttt{*} and \texttt{/}, the \textbf{and} and \textbf{or} operations are not in the same group.)
