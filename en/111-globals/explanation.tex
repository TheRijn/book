\paragraph{Globale variabelen}

Variabelen die al bestaan kan je ook gebruiken binnen functies. Deze variabelen noemen we \emph{globale variabelen}. De waarde van de variabele binnen de functie is afhankelijk van wanneer de functie wordt aangeroepen. Kijk bijvoorbeeld eens naar het volgende stukje code:

\begin{verbatim}
word = "dolphin"
void say():
  print(word)
say()
word = "cat"
say()
\end{verbatim}

Dit stukje code print eerst \texttt{dolphin} en daarna \texttt{cat}. Ten tijde van de eerste aanroep naar de functie \texttt{say} heeft de variabele \texttt{word} de waarde \texttt{"dolphin"}. Dus wordt er eerst \texttt{dolphin} geprint. Daarna heeft de variabele \texttt{word} de waarde \texttt{"cat"}. De tweede aanroep print daarom \texttt{cat}.

\paragraph{Variabelen aanpassen} Een functie kan ook de variabelen buiten de functie aanpassen. Net zoals in dit stukje code:

\begin{verbatim}
number = 0
void increment():
  number += 1
increment()
increment()
increment()
print(number)
\end{verbatim}

Het bovenstaande stukje code print de waarde \texttt{3}, want na het driemaal aanroepen van de functie \texttt{increment} is de waarde van \texttt{number} verhoogd naar 3.
