\paragraph{When the condition fails}

Below an \texttt{if} statement you might find an \texttt{else} statement. The two statements are then connected, and the \texttt{else} statement defines what should happen in case the condition in the \texttt{if} statement yields \texttt{false}.

\begin{nnflisting}
a = 1
if(a < 0)
    a = a + 10
else
    a = a - 10
\end{nnflisting}

Tracing the program above would look like this:

\begin{tracelist}[lccccccc]
 & \texttt{1} & \texttt{2} & \texttt{4} & \texttt{5} \\ \hline
\\[-1em]
\texttt{a} & \fbox{\texttt{1}} & \texttt{1} & \texttt{1} & \texttt{-9}
\end{tracelist}

In summary, an \texttt{if}-\texttt{else} combination always describes two possibilities: what should happen in case the condition is met (\texttt{a < 0}), and what should happen in case the condition is not met, \texttt{a} having a value that meets the opposite condition \texttt{a >= 0}.

\paragraph{Connecting multiple conditions}

To describe even more than two options, \texttt{if} statements can be augmented by adding \texttt{else\,if} statements. In the next fragment, we specify three options: \texttt{a < 0}, \texttt{a > 0} and otherwise \texttt{a = 0}. For each of the three cases, a single dependent instruction is specified, which will only be executed as that specific condition is met. All other cases are then \underline{skipped}.

\begin{nnflisting}
a = 3
if(a < 0)
    a = -1
else if(a > 0)
    a = 1
else
    a = 100
\end{nnflisting}

The program starts with \texttt{a} being the integer \texttt{3}, and the trace looks like this:

\begin{tracelist}[lccccccc]
 & \texttt{1} & \texttt{2} &  \texttt{4} & \texttt{5} &  \\ \hline
\\[-1em]
 \texttt{a} & \fbox{\texttt{3}} & \texttt{3} & \texttt{3} & \fbox{\texttt{1}} \\
\end{tracelist}

By using even more \texttt{else\,if} statements, you can specify further options. The \texttt{else} statement always comes last, because it describes ``any other possibility''.
